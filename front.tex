%% ------------------------------ Abstract ---------------------------------- %%
\begin{abstract}
Precision medicine aims to improve disease interventions by incorporating the variability of patients. Clinicians often need to make sequences of decisions for patients with chronic conditions, such as diabetes, cancer, HIV, etc. This sequential decision making problem in precision medicine is mathematically formalized as a dynamic treatment regime (DTR). DTRs are defined as a sequence of decision rules, one for each decision point, that take patient cumulative information as input and output recommended treatment assignments. In most cases, a treatment regime is considered to be optimal, if it optimizes the expected value of a single scalar potential outcome in a population of interest. However, this framework neglects the practical clinical need of balancing several competing outcomes such as, treatment effectiveness, side effect burden, cost, etc. To handle the trade-off among multiple competing outcomes, we propose a new framework where the primary potential outcome of interest is optimized, subject to constraints on secondary outcomes. In Chapter 1, we introduce dynamic treatment regimes, and develop a new method to construct a constrained optimal regime with a single decision point. In Chapter 2, we extend the method into the multiple decision point setting. In Chapter 3, we consider the infinite horizon setting, which is suitable for life-long clinical conditions, and discuss some potential future research directions.
\end{abstract}


%% ---------------------------- Copyright page ------------------------------ %%
%% Comment the next line if you don't want the copyright page included.
\makecopyrightpage

%% -------------------------------- Title page ------------------------------ %%
\maketitlepage

%% -------------------------------- Dedication ------------------------------ %%
\begin{dedication}
 \centering To my parents.
\end{dedication}

%% -------------------------------- Biography ------------------------------- %%
\begin{biography}
The author was born in Shanghai, China. She obtained her bachelor’s degree in Biotechnology from Fudan University in Spring, 2011. Her thesis research has been focusing on dynamic treatment regimes in the precision medicine paradigm, under the guidance of her advisor Dr. Eric Laber. She is broadly interested in science and technology, and is possessed with an insatiable curiosity. Her interests include, but are not limited to, statistics, bioinformatics, machine learning, artificial intelligence, quantum computing and so on. She will graduate with her doctoral degree in Statistics in December, 2017. She has also earned  masters of Bioinformatics and Statistics while pursuing her doctoral degree.
\end{biography}

%% ----------------------------- Acknowledgements --------------------------- %%
\begin{acknowledgements}
I would like to thank my advisor and committee members for their guidance and support. I would like to thank professors, staff and friends from the Department of Statistics and the Bioinformatics Research Center at North Carolina State University for their endless support along the way. I also would like to thank my family for their understanding and encouragement.
\end{acknowledgements}


\thesistableofcontents

\thesislistoftables

\thesislistoffigures
   